\documentclass{article}
\usepackage[utf8]{inputenc}
\usepackage{parskip}
\usepackage[fencedCode]{markdown}
\usepackage[dutch]{babel}
\usepackage[a4paper,margin=3cm]{geometry}

\title{%
    Verslag Eindopdracht Programmeren \\
    \large Opdracht: Cellulaire Automaten}

\author{Isabel Baas, Daan Braakman \& Mylan Koobs}
\date{Januari 2023}

\begin{document}

\maketitle
\newpage

\tableofcontents
\newpage
\section{Inleiding}
Cellulaire automaten bestaan uit een raster van zogenoemde cellen die allemaal een eindig aantal toestanden aan kunnen nemen. Een aantal voorbeelden van zo'n toestand zouden kunnen zijn: levend, dood, ziek of eenzaam. Een cel neemt een toestand aan op basis van vooraf opgestelde regels en interactie met zijn buurcellen. Deze regels worden dan toegepast op de huidige toestand van deze cel. Op deze manier kan in de verloop van de tijd een systeem worden gesimuleerd waar dan weer vakgerichte conclusies uit getrokken kunnen worden. Cellulaire automata worden niet alleen gebruikt in de wiskunde, maar ook in de biologie bijvoorbeeld. Zo wordt in de biologie cellulaire automata gebruikt om koraalkolonies te simuleren. 

De geschiedenis van cellulaire automata begint rond de jaren 50. De wiskundige John von Neumann wilde namelijk een machine maken die zichzelf na kon bouwen, dit wilde hij bereiken door zelfreproductie uit de natuur na te bootsen. Het eerste cellulaire automaat wat hij heeft gemaakt bevatte 29 verschillende toestanden en veel verschillende regels die verschillende mechanismes konden nabootsen. Dit was het eerste cellulaire automaat dat echt gecreëerd was.

Voor deze eindopdracht moeten wij zelf zo'n cellulaire automata nabootsen door gebruik te maken van python. De basis van de opdracht begint bij het coderen van een raster met de cellen die zo'n toestand aan kunnen nemen. Dit moet zichtbaar worden gemaakt door middel van bijvoorbeeld een computerprogramma. Vanuit hier kunnen dan regels worden geschreven zodat het systeem gesimuleerd kan worden. 

Het lastige aan deze opdracht is dat er veel verschillende basiszaken zijn die met elkaar gecombineerd moeten worden. De cellen zijn namelijk afhankelijk van het raster, maar ook van de regels die je hebt geschreven. Op deze manier is er dus veel interactie tussen de verschillende stukjes code waardoor een klein foutje er al snel voor kan zorgen dat de rest van je code ook niet meer werkt. Een kleine aanpassing in je code kan dus een voor een grote aanpassing in de rest van je code zorgen. 
\section{Algoritme}


\subsection{Structuur} %nog rede voor de keuzes?
In de code zitten meerdere klassen. De meest algemene klasse, ...., heeft een rooster met een zelf te kiezen aantal cellen in iedere dimensie. De dimensie is ook zelf te kiezen. Voor het rooster is er gebruik gemaakt van numpy arrays. Deze maken het makkelijker om te werken met meerdere dimensies. \newline
Daarnaast heeft de klasse een regel-functie, die zelf gedefinieerd kan worden, bepaald de toestanden van de volgende generaties cellen. Het is op deze manier gedaan zodat het eenvoudig is om zelf regels te definiëren. \newline Ook bevat de klasse een functie voor het updaten, deze wordt later verder toegelicht.

Er zijn twee klassen afgeleid van de algemene klasse, die voor één- en voor tweedimensionale CA's. Deze bevatten alles van de algemene klasse en dat wat er nodig is om ze grafisch weer te geven. Er is een neighborhood klasse met functies waarmee de staten van omliggende cellen van een cel mee kunnen worden opgehaald. Deze kunnen gebruikt worden bij het definiëren van de regel-functies om eenvoudig de staten van buurtcellen op te halen. 

\subsection{Update}
De update-functie update het gehele rooster door voor elke cel de rules functie op te roepen om te bepalen wat zijn nieuwe staat moet zijn. De complexiteit hangt af van de complexiteit van de regel-functie. Deze noemen we $m$. Bij $n$ cellen wordt de rules-functie $n$ keer aangeroepen, dus dan is de complexiteit $O(mn)$.

\subsection{Grafische weergave}
De grafische weergave voor de 2D CA's tekent elke cel in het grid en doet verder niks wat afhangt van input, dus zij het aantal cellen $n$, dan is de complexiteit $O(n)$.

Voor de grafische weergave van de eendimensionale CA was het nodig om als het waren mee te scrollen met de nieuwe generaties. Deze raakten namelijk na een aantal generaties buiten beeld. Om dit probleem op te lossen, wordt elke nieuwe generatie toegevoegd aan een lijst. Zodra de lijst meer generaties bevat dan grafisch weer te geven zijn op het scherm, word de oudste generaties uit de lijst gehaald en is deze dus ook niet meer te zien.

Zij het aantal cellen in de CA $n$ en het aantal generaties in de lijst (dus het aantal generaties dat getekend wordt) $g$, dan is de tijdscomplexiteit dus $O(nm)$.

\subsection{Neighborhood}
The neighborhood functies zijn bedoeld om bij het maken van regels-functies makkelijk toegang te krijgen tot de toestanden van buurtcellen. Er wordt gebruikt gemaakt van een Moore-Neighborhood met een zelf te kiezen afstand, die we $r$ noemen. Ook kan dit voor elk aantal dimensies, die we $d$ noemen (de algemene functie maakt wel gebruik van recursie en is daarom wat minder snel, dus voor 1- en 2-dimensionale CA's kan beter gebruik worden gemaakt van de daarvoor bestemde neighborhood-functies). 

De functies lezen de toestanden van in totaal $r^d$ cellen, en aangezien het lezen in constante tijd kan, is de uiteindelijke complexiteit ook $O(r^d)$.

Voor de periodieke neighborhood-functie wordt nog gebruik gemaakt van modulo-operaties (modulo scherm-hoogte of -breedte), dus bij grote schermgroottes kan dit voor extra complexiteit zorgen doordat de complexiteit van modulo-operaties dan groter wordt. 

\subsection{Overig}
Als laatste zijn er nog wat functies die cellen kunnen invullen, zoals \verb|setcells|. Deze gaan elke cel bij langs die wordt aangegeven (of bij sommige functies zoals \verb|random|, alle cellen), en geeft hem een bepaalde toestand (wat in constante tijd kan). Zij het aantal in te vullen cellen gelijk aan $n$, dan is de tijdscomplexiteit van deze functies dus ook $O(n)$.

\section{Handleiding}
\begin{markdown}


## Importeren
Stop het CA.py bestand in dezelfde map als het project, en importeer het als volgt:
```python
import CA
#als de naam van het bestand veranderd is, gebruik dan de nieuwe naam ipv CA
```
Zorg er ook voor dat numpy geïnstalleerd is, want dat is wat gebruikt wordt voor het grid van de CA. Voor het gebruik maken van de ingebouwde visualisatie zal ook pygame geïnstalleerd moeten zijn.

## Aanmaken van CA's

Verschillende cellulaire automaten kunnen gemaakt worden met behulp van de basisklasse `CellularAutomata`. Een CA kan als volgt gemaakt worden:

`myCA = CA.CellularAutomata(shape, rules)`

waar shape de vorm van de CA is, wat weergegeven word als een lijst met het aantal cellen in elke dimensie, en rules de functie is die, gegeven een cell, de index van deze cell en de rest van het rooster, de geüpdatete staat van deze cell geeft. 

### Voorbeeld 1

We maken hier een 3-dimensionale CA met in elke dimensie 10 cellen, die als regel heeft dat als een cell de waarde 1 heeft, hij 0 wordt en andersom.

We definiëren eerst de functie voor onze regel, en vullen die in wanneer we de CA maken. 
```python
def invert(cell, index, grid):
    if cell == 0:
        return 1
    else:
        return 0

myCA = CA.CellularAutomata([10,10,10], invert)
```
### 1- en 2-dimensionale CA's

Ook kunnen we eenvoudig 1- en 2-dimensionale CA's maken met de  `Cellular1D` en `Cellular2D` klassen. Deze bevatten al ingebouwde functies voor visualisatie, wat handig kan zijn.

Een 1-dimensionale CA met 10 cellen kan als volgt worden aangemaakt:
```python
#we gebruiken dezelfde functie invert als bij het eerste voorbeeld
myCA_1D = CA.Cellular1D(10, invert)
```
Merk op dat hier alleen een getal gegeven wordt voor de grootte, en geen lijst. 

Een 2-dimensionale CA met 10 cellen in de breedte en 5 in de hoogte kan als volgt worden gemaakt:
```python
myCA_2D = CA.Cellular2D(10, 5, invert)
```
Ook hier wordt de breedte en hoogte los gegeven, en niet in een lijst. 

## Het invullen van cellen

Wanneer een CA wordt aangemaakt, begint hij met 0 in elke cell. Er zijn verschillende manieren om waardes aan cellen toe te kennen. 

De meest algemene, die voor elke CA werkt, is `setcells`. Deze functie neemt een lijst met index-tuples en een bepaalde waarde, en kent voor elke index-tuple de bijbehorende cell de gegeven waarde toe. 

```python
#op plaatsen (5,5,5), (0,0,0) & (1,2,3) wordt de waarde van de cell 1. 
myCA.setcells([(5,5,5), (0,0,0), (1,2,3)], 1)
```

Ook bestaat er een functie die alle waarden terugzet naar 0, `setzeros`. Deze werkt ook voor elke CA.
```python
myCA.setzeros()
```
Daarnaast bestaat er ook de functie `random`, die ook voor elke CA werkt, en die alle cellen random 0 of 1 maakt. 
```python
myCA.random()
```
Verder is er ook een functie speciaal voor `Cellular1D`, namelijk `start_middle`, die de cell in het midden naar 1 verandert. 
```python
myCA_1D.start_middle()
```

## Updaten van CA

Het updaten van CA's kan eenvoudig gedaan worden met de `update` functie. Deze functie roept voor iedere cell de rules-functie op, en geeft de cell de output als waarde. 

```python
myCA.update()
```

Ook kan de CA automatisch een gegeven aantal keer geüpdate worden door middel van de `run` functie. Deze neemt een getal voor het aantal keer dat de CA geüpdate moet worden, en update hem dan zoveel keer.

```python
myCA.run(500)
```

## Grafische weergave

Voor de 1- en 2-dimensionale CA klassen bestaan er ingebouwde functies voor het visualiseren. 

Allereerst bestaat voor beide de `draw` functie. Deze neemt als argumenten een screen (een pygame Surface), een cellsize voor de grootte van de cellen (in pixels), en een surflist, een lijst van pygame Surfaces. 

Het `screen` is hetgeen waar alles op getekend word, en `surflist` is een lijst met surfaces met verschillende kleuren, die gebruikt worden om alle cellen in te kleuren. Hierbij wordt de toestand van de cell gebruikt als index voor de `surflist`, dus een cell met toestand in wordt ingekleurd met de surface op plek 0 van de lijst, enzovoort. Zorg ervoor dat deze surfaces vierkant zijn met dezelfde cellgroote als de `draw` functie.

```python
screen = pygame.display.set_mode((640,640))
wit = pygame.Surface(10,10)
wit.fill((255,255,255))
zwart = pygame.Surface(10,10)
zwart.fill((0,0,0))
surflist = [zwart, wit]

#Dit kan ook met 1D CA's
myCA_2D.draw(screen, 10, surflist)
```

Daarnaast bestaan er een functie die het updaten en weergeven automatisch doen, `runvisual`. Deze functie neemt als argumenten de breedte en hoogte van het scherm (in pixels), de `changetime` (tijd tussen updates in millisecondes, vooral nuttig bij CA's die anders te snel gaan), de cellsize in pixels, en een lijst met kleuren voor elke toestand. Als deze lijst kleiner is dan het aantal toestanden, krijgen cellen met een toestand waarvoor geen kleur in de lijst zit de laatste kleur uit de lijst. 

De visualisatie kan op pauze gezet worden door op de spatiebalk te drukken.

```python
#dit kan wederom ook weer met 1-dimensionale CA's
myCA_2D.runvisual(640,640, 100, 10, [(0,0,0), (255,255,255)])
```

## Het maken van regels

Voor het maken van regels kan gebruik gemaakt worden van ingebouwde functies die de staten van buurtcellen ophalen. Deze zitten in de `Neighborhoods`-klasse. Er zijn functies voor 1- en 2-dimensionale buurten, en functies voor buurten met willekeurige dimensie. Voor 1- en 2D wordt aangeraden om de daarvoor bestemde functies te gebruiken en niet de algemene, aangezien die meestal iets langzamer is doordat er gebruik gemaakt wordt van recursie. Er bestaan voor elk van die categorieën nog twee soorten functies, waar het verschil zit in de randvoorwaarden. 

De standaard `Neighborhoods.get_neighbors` functie neemt naast parameters voor het grid, de index van de huidige cell en de reach ook nog een waarde genaamd `default` aan, die gebruikt wordt wanneer een cell buiten het grid ligt (en dus eigenlijk niet bestaat). 

De `Neighborhoods.get_neighbors_periodiek` gaat verder vanaf de andere kant wanneer cellen niet bestaan.

Deze functies geven als output een lijst met de staten van de cellen. De volgorde begint bij de laagste indexen in alle dimensies, beweegt naar de hoogste index in de hoogste dimensie, en beweegt dan 1 omhoog in de één-na-hoogste dimensie en gaat weer naar de hoogste index in de hoogste dimensie, en dit gaat zo door totdat alle cellen zijn opgehaald.

Als voorbeeld maken we hier een functie voor rule 22, één van de elementaire CA's. 

```python
import CA

def rule22(cell, idx, grid):
    #hier halen we de staten van de buurcellen op
    states = CA.Neighborhoods.get_neighbors1D_periodiek(grid, idx, 1)

    #definieren links, rechts en het midden voor gemak
    left = states[0]
    center = states[1]
    right = states[2]
    
    #alle voorwaarden met de bijbehorende output 
    # (gebruik makend van dat 0 False geeft bij gebruik als boolean
    # en getallen ongelijk 0 True)
    if left and center and right:
        return 0
    elif left and center and not right:
        return 0
    elif left and not center and right:
        return 0
    elif left and not center and not right:
        return 1
    elif not left and right and center:
        return 0
    elif not left and center and not right:
        return 1
    elif not left and not center and right:
        return 1
    else: 
        return 0
```
Ook kunnen we hiermee vrij eenvoudig een regelfunctie maken voor Game of Life. 

```python
def Game_of_life_rules(cell, idx, grid):
        states = CA.Neighborhoods.get_neighbors2D_periodiek(grid, idx, 1)
        levende_buren = 0
        #telt de levende buren 
        for i in states: 
                if i == 1:
                    levende_buren = levende_buren + 1
        
        #laat er een geboren worden als er precies 3 levende buren zijn          
        if cell == 0: 
            if levende_buren == 3: 
                return 1
            else:
                return 0
        

        #laat een levende cel sterven door over- of onderbevolking   
        if cell == 1: 
            #er is hier rekening gehouden met dat cell ook leeft
            if levende_buren > 4 or levende_buren < 3: 
                return 0
            else:
                return 1
```

Ook kunnen we gebruik maken van meerdere toestanden, wat te zien is in het volgende voorbeeld waarbij zieke cellen zijn toegevoegd aan Game of Life, die andere cellen ziek kunnen maken.

```python
def Game_of_Life_Sickness(cell, idx, grid):
    states = Neighborhoods.get_neighbors2D_periodiek(grid, idx, 1)
    levende_buren = 0
    zieke_buren = 0
    #telt de levende en zieke buren 
    for i in states: 
        if i == 1:
            levende_buren = levende_buren + 1
        if i >= 2:
            zieke_buren = zieke_buren + 1

    #laat er een geboren worden als er precies 3 levende buren zijn      
    if cell == 0:
        if levende_buren == 3:
            return 1
        else:
            return 0


    if cell == 1:
        #maakt een cel ziek (toestand 2) als er 3 of meer zieke buurcellen zijn
        if zieke_buren >= 3:
            return 2
        #laat een levende cel sterven door over- of onderbevolking
        elif levende_buren > 4 or levende_buren < 3: 
            return 0
        else:
            return 1


    if cell >= 2 and cell < 5:
        #als een zieke cell meer dan 2 levende buren heeft, geneest de cell
        if levende_buren > 2:
            return 1
        #anders wordt de cell nog zieker
        else: 
            return cell+1
    
    #als de cell erg ziek is, gaat hij dood
    if cell >= 5:
        return 0
```

## Overzicht van functies

`CA.CellularAutomata(shape: list, rules)`: Maakt een cellulaire automata aan met de shape voor de vorm van het grid, en een functie rules met als input een cell, een index en het grid en als output de geupdate staat van de cell.

`CA.Cellular1D(size: int, rules)`: Maak een 1-dimensionale CA met grootte size en een functie rules met als input een cell, een index en het grid en als output de geupdate staat van de cell.

`CA.Cellular2D(width: int, height: int, rules)`: Maak een 2-dimensionale CA met width cellen in de breedte en height cellen in de hoogte en een functie rules met als input een cell, een index en het grid en als output de geupdate staat van de cell.

`CA.CellularAutomata.update()`: Update het hele grid door voor elke cel de rules functie op te roepen om te bepalen wat zijn nieuwe staat moet zijn.

`CA.CellularAutomata.run(updates: int)`: Update het grid een gegeven aantal keer.

`CA.CellularAutomata.setcells(coordinates: list, value: int)`: Verandert de waarden van cellen met de gegeven coördinaten naar de gegeven waarde.

`CA.CellularAutomata.setzeros()`: Verandert alle toestanden in het grid naar 0.

`CA.CellularAutomata.random(max: int)`: Verandert alle toestanden in het grid naar random getallen tussen 0 en max.

`CA.Cellular1D.start_middle()`: Maakt de toestand van de middelste cel gelijk aan 1. Vooral bedoelt voor de elementaire CA's.

`CA.Cellular1D.draw(screen: pygame.Surface, cellsize: int, surflist: list)`: Tekent het grid op een screen, scrollt automatisch mee als onderkant van scherm wordt bereikt. Gebruikt de surflist met surfaces om de cellen in te tekenen op basis van de toestand.

`CA.Cellular1D.runvisual(width: int, height: int, changetime: int, cellsize: int, colorlist: list)`: Start pygame visualisatie met bepaalde width en height in pixels, changetime geeft tijd in ms tussen updates en colorlist zorgt voor de kleuren per toestand (als lijst met RGB-waardes).

`CA.Cellular2D.draw(screen: pygame.Surface, cellsize: int, surflist: list)`: Tekent het grid op een screen. Gebruikt de surflist met surfaces om de cellen in te tekenen op basis van de toestand.

`CA.Cellular2D.runvisual(width: int, height: int, changetime: int, cellsize: int, colorlist: list)`: Start pygame visualisatie met bepaalde width en height in pixels, changetime geeft tijd in ms tussen updates en colorlist zorgt voor de kleuren per toestand (als lijst met RGB-waardes).

`CA.GameOfLife(width: int, height: int)`: Maakt een 2D CA aan met Game of Life regels.

`CA.GameOfLife.glider(offset_width: int, offset_height: int, direction: int)`: Zet een zogenaamde glider in het grid, met de offset ten opzichte van linksboven en de richting een getal tussen 0 en 3, met 0 voor naar linksboven, 1 voor naar rechtsboven, 2 voor linksonder en 3 voor rechtsonder.

`CA.Neighborhoods.get_neighbors_1D(grid: np.ndarray, idx: list, reach: int, default: int)`: geeft een list met de toestanden van buurtcellen in een 1D Moore-Neighborhood met reach voor de grootte van de buurt. Gebruikt default als cellen niet bestaan.

`CA.Neighborhoods.get_neighbors_1D_periodiek(grid: np.ndarray, idx: list, reach: int)`: geeft een list met de toestanden van buurtcellen in een 1D Moore-Neighborhood met reach voor de grootte van de buurt. Gaat verder vanaf de andere kant als cellen niet bestaan.

`CA.Neighborhoods.get_neighbors_2D(grid: np.ndarray, idx: list, reach: int, default: int)`: geeft een list met de toestanden van buurtcellen in een 2D Moore-Neighborhood met reach voor de grootte van de buurt. Gebruikt default als cellen niet bestaan.

`CA.Neighborhoods.get_neighbors_2D_periodiek(grid: np.ndarray, idx: list, reach: int)`: geeft een list met de toestanden van buurtcellen in een 2D Moore-Neighborhood met reach voor de grootte van de buurt. Gaat verder vanaf de andere kant als cellen niet bestaan.

`CA.Neighborhoods.get_neighbors(grid: np.ndarray, idx: list, reach: int, default: int)`: geeft een list met de toestanden van buurtcellen in een willekeurig-dimensionale Moore-Neighborhood met reach voor de grootte van de buurt. Gebruikt default als cellen niet bestaan.

`CA.Neighborhoods.get_neighbors_periodiek(grid: np.ndarray, idx: list, reach: int, default: int)`: geeft een list met de toestanden van buurtcellen in een willekeurig-dimensionale Moore-Neighborhood met reach voor de grootte van de buurt. Gaat verder vanaf de andere kant als cellen niet bestaan.
\end{markdown}

\newpage
\section{Discussie}

\subsection{Samenvatting} %beter refereren
In dit verslag is alles te lezen over de geschreven code. Aller eerst is in de inleiding te lezen waar de code voor is bedoeld en wordt de achtergrond van het onderwerp besproken. \newline
Er is een algemene klasse ontworpen voor CA's. Daarnaast is er de mogelijkheid om één- en tweedimensionale CA's grafisch weer te geven met een paar sets aan regels, zoals rule22 voor 1D en de regels van Game Of Life voor 2D. Ook is er de mogelijkheid om te kijken naar meer dan 2 toestanden. Naast dode en levende cellen, zijn er dan ook zieke cellen.\newline 
Hoe de code hiervoor in elkaar steekt is te lezen in de bespreking van het algoritme.\newline
Ook is er een uitgebreide handleiding voor het gebruik van de code, met voorbeelden.\newline
Als laaste worden de tekortkomingen van de code bespoken.

\subsection{Tekortkomingen code}
Iets wat op ons opviel tijdens het afronden van de eendimensionale klasse is dat het “tekenen” van iedere nieuwe rij in de grafische weergaven steeds langzamer verliep naarmate het programma langer had gerund. Na het controleren bleek python ook een disproportioneel groot deel van de processor te gebruiken. Dit is een probleem waar we tegenaan liepen. Het probleem viel ons op toen we de code hadden toegevoegd die er voor zorgt dat wanneer de grote van de CA groter was dan het scherm, de grafische weergave wel in het midden van het scherm begon en dat de CA als het ware buiten het scherm verder loopt. Echter leek dit niet de oorzaak te zijn. \newline
Een vergelijkbaar probleem kwam omhoog nadat de twee dimensionale klasse af was. Een CA met de regels van Game of life en een rooster grote van bijvoorbeeld 50 bij 50 gaat zo snel als is ingesteld in de \emph{changetime} optie. Maar hoe groter het rooster wordt ingesteld, hoe langzamer de CA gaat. Er is geen directe oorzaak gevonden, niet iets kleins wat opgelost zou kunnen worden zonder dat een groot deel van de code herschreven zou moeten worden. Aangezien we dit probleem pas laat tegenkwamen is dit ook niet opgelost.

De volgende punten zijn niet zo zeer tekortkomingen, maar meer toevoegingen die we graag hadden gedaan. \newline
De eerste is de mogelijkheid om als het ware over het rooster heen te kunnen bewegen wanneer het rooster groter is dan het scherm. Aangezien dit niet een vereiste was voor de opdracht en het waarschijnlijk erg veel werk zou vergen, hebben we besloten dit achterwegen te laten.\newline
Daarnaast wilde we het graag mogelijk maken om cellen tot leven te brengen door er op te klikken, naast dat je geselecteerde cellen een toestand kan geven met de \emph{setcells} optie. Zo zou het veel makkelijker zijn om bijvoorbeeld een glider gun te maken in Game of life. Het begin van het maken van deze optie is de mogelijk om de CA op pauze te zetten.

\subsection{Taak verdeling \& tijdsmanagement}

De taakverdeling was iets dat heel natuurlijk verliep. Buiten de werkcolleges werd er apart gewerkt aan onderdelen van de code en het verslag en in de werkcolleges werkte we veel samen. Vaak dachten we samen aan oplossingen voor een onder deel wat op dat moment relevant was en typte een persoon het uit in code. De onderdelen van het verslag zijn iets bewuster verdeeld, maar ook dit ging heel simpel. Uit eindelijk is al het werk dat nodig was voor het project niet compleet evenredig verdeeld, maar niet in mate dat het voor iemand een probleem was. Het schrijven van code gaat nou eenmaal makkelijker voor mensen met meer ervaring. Als we eerder waren begonnen was er wellicht ruimte geweest om nog meer van elkaar te leren. \newline
Tijd was wel iets wat lichte stres heeft opgeleverd. Dit kwam omdat we te laat begonnen omdat we dachten dat het project pas aan het einde van de tentamen week ingeleverd moest worden. Ondanks dat hebben we er niet echt last van gehad.

\end{document}
